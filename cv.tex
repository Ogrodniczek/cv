\documentclass[11pt,a4paper,sans]{moderncv}

\usepackage[nodayofweek]{datetime}
\usepackage[utf8]{inputenc}
\usepackage[scale=0.80]{geometry}

\newcommand*{\skypesymbol} {%
  \protect\raisebox{-0.085em}{%
\protect\begin{tikzpicture}[y=0.08em,x=0.08em,xscale=0.022,yscale=-0.022, inner sep=0pt, outer sep=0pt]
\protect\path[fill=color2,even odd rule] (487.6550,288.9690) .. controls (489.0610,278.5690) and
  (489.8700,267.9960) .. (489.8700,257.2330) .. controls (489.8700,128.0770) and
  (384.5990,23.3610) .. (254.7670,23.3610) .. controls (241.8630,23.3610) and
  (229.2120,24.4210) .. (216.9010,26.4410) .. controls (194.8280,12.0570) and
  (168.5590,3.6740) .. (140.2880,3.6740) .. controls (62.7660,3.6740) and
  (0.0000,66.4820) .. (0.0000,143.9800) .. controls (0.0000,172.1780) and
  (8.2990,198.3740) .. (22.5900,220.3690) .. controls (20.6650,232.3860) and
  (19.6810,244.6920) .. (19.6810,257.2290) .. controls (19.6810,386.4050) and
  (124.8980,491.1100) .. (254.7660,491.1100) .. controls (269.4230,491.1100) and
  (283.6930,489.6840) .. (297.5620,487.1780) .. controls (319.1120,500.5470) and
  (344.4960,508.3260) .. (371.7080,508.3260) .. controls (449.2100,508.3260) and
  (512.0010,445.5020) .. (512.0010,368.0120) .. controls (511.9980,338.7190) and
  (503.0410,311.4840) .. (487.6550,288.9690) -- cycle(276.7400,429.5960) ..
  controls (202.0340,433.4870) and (167.0750,416.9590) .. (135.0500,386.9050) ..
  controls (99.2850,353.3370) and (113.6520,315.0500) .. (142.7900,313.1040) ..
  controls (171.9120,311.1590) and (189.3980,346.1160) .. (204.9410,355.8400) ..
  controls (220.4650,365.5280) and (279.5340,387.6000) .. (310.7350,351.9320) ..
  controls (344.7100,313.1040) and (288.1410,293.0120) .. (246.6760,286.9300) ..
  controls (187.4730,278.1640) and (112.7260,246.1370) .. (118.5410,183.0230) ..
  controls (124.3580,119.9490) and (172.1230,87.6090) .. (222.3910,83.0470) ..
  controls (286.4680,77.2300) and (328.1820,92.7540) .. (361.1760,120.9070) ..
  controls (399.3270,153.4360) and (378.6840,189.8010) .. (354.3770,192.7270) ..
  controls (330.1660,195.6360) and (302.9730,139.2230) .. (249.5860,138.3750) ..
  controls (194.5590,137.5110) and (157.3690,195.6360) .. (225.3000,212.1590) ..
  controls (293.2660,228.6640) and (366.0500,235.4450) .. (392.2610,297.5760) ..
  controls (418.4900,359.7130) and (351.5070,425.7010) .. (276.7400,429.5960) --
  cycle;
\protect\end{tikzpicture}}%
  ~}%

\usepackage{catchfile}
\newcommand{\getenv}[2][]{%
 \CatchFileEdef{\temp}{"|kpsewhich --var-value #2"}{\endlinechar=-1\relax}%
 \if\relax\detokenize{#1}\relax\temp\else\edef#1{\temp}\fi%
}%

\newcommand*{\pin}{%
    \includegraphics[height=\heightof{M}]{pin}
}
\newcommand{\newpara}{
        \vskip 0.2cm
    }

\moderncvstyle{classic}
\moderncvcolor{blue}
\nopagenumbers{}
\getenv[\lang]{lang}

\name{Maciej}{Ogrodniczek}
\address{\pin Wrocław, Poland}
\phone[mobile]{+48~607~666~399}
\email{maciej@ogrodniczek.com}
\social[linkedin]{maciejogrodniczek}

\begin{document}
\makecvtitle
DevOps Engineer focused on cloud computing, automation of software integration, Python programming, system administration, delivery process and configuration management. \\
Passionate about containers and Kubernetes.
\newpara
Key technologies: Python, Linux, AWS, Jenkins, Ansible, Docker, Kubernetes.

\section{Experience}
\cventry{2018.01--2018.07}{Senior DevOps Software Engineer}{Softserve}{}{}{
    Worked as DevOps in the Data Ingestion team in a Danish company working in big data field.
\begin{itemize}
    \item Migrated services deployed manually to code using Chef and Terraform.
    \item Supported and maintained multiple Nifi instances for the development team.
\end{itemize}
\newpara
Technologies: Chef, AWS, Terraform, Nifi, Rancher, Hadoop.
}

\cventry{2017.03--2017.12}{Senior Software Engineer}{Intive}{}{}{
    Goal of this project was to automate tasks that were performed manually, like building process, release note generation basing on multiple sources etc. Second part of the project was to perform DevOps activities (CI and Docker related) for a micro-serviced application.
\begin{itemize}
    \item Automation of release process: Used multiple Python libraries to achieve the goal
    \item Improvements for CI: Docker storage optimization, Gerrit's pre-submit triggers improvements, introduced dynamic Jenkins slaves based on AWS.
\end{itemize}
\newpara
Technologies: Python, Jenkins, Docker, Gerrit, AWS.
}

\cventry{2015.10--2016.10}{Technical Leader, SCM}{Nokia Solutions and Networks}%
    {}{}{
Worked as a part of SCM Team responsible for releasing/maintaining/compiling software for Nokia network nodes
hardware. I was assigned to Core Team, where we were creating concepts and
implementing them.
\begin{itemize}
    \item Designing new features and tasks planning for SCM Team,
    \item Creating own Nokia Distribution using Yocto Project,
    \item Helping SCM Team members in troubleshooting problems in wide spectrum,
    \item Kubernetes proof of concept in CI environment,
    \item Assuring HA for CI environment,
    \item Maintaining build environment and DNS servers for SCM purposes,
    \item Secured and hardened CI environment for crucial client deliveries.
\end{itemize}
\newpara
Technologies: Yocto, Python, Eucalyptus (EC2, S3), Jenkins, Docker, Kubernetes.
}

\cventry{2013.02--2015.09}{SCM Engineer}{Nokia Solutions and Networks}%
    {}{}{
Worked as a part of LTE SCM Team responsible for releasing/maintaining/compiling software for LTE branch of Nokia
products. I was assigned to Central Build Team, where we were responsible for a
part of CI where the final product is created. Additionally, we were responsible for
compilation of the product.
\begin{itemize}
    \item Work with large scale CI ecosystem,
    \item Refactoring and development of new tools used in SCM,
    \item Introducing good practices in development, like TDD and OOP,
    \item Maintaining and tuning automatic building processes,
    \item Deployed Eucalyptus Cloud infrastructure in a build system,
    \item Creating and implementing new parts of Continuous Integration process,
    \item Provided multiple Python trainings for ca. 200 people inside Nokia and also Python trainings at Wrocław University of Technology.
\end{itemize}
\newpara
Technologies: Python, Eucalyptus (EC2, S3), Jenkins, Bash, Make.
}

\cventry{2012.03--2013.01}{Junior Software Engineer}{Tieto Poland}%
    {}{}{
        Worked as a part of an external Team for a Client from telecommunications industry.
\begin{itemize}
    \item Refactoring and development of new tools used in SCM,
    \item Implementing new control processes to software release process,
    \item Development and maintenance of building automation tools,
    \item Produce builds of products.
\end{itemize}
\newpara
Technologies: Bash, Jenkins, Python.
}

\section{Education}
\cventry{2007--2013}{Bachelor of Science degree}{Wroclaw University of Technology}{}{}%
    {Faculty: Electronics, Department: Teleinformatics \\
    Thesis: “A design of a computer network for the hosting corporation”}

\section{Languages}
\cvitemwithcomment{English}{professional working proficiency}{}
\cvitemwithcomment{Polish}{native}{}

\section{Skills}
\cvitem{Administration}{Linux - RHEL and Debian based distributions, CoreOS.}
\cvitem{Scripting}{Python (advanced), Bash.}
\cvitem{CVS}{GIT, Subversion.}
\cvitem{CM}{Ansible, Chef.}
\cvitem{CI}{Jenkins, TravisCI.}
\cvitem{Cloud computing \& virtualization}{
    \begin{itemize}
        \item Amazon Web Services, Eucalyptus, RIAK (S3),
        \item Kubernetes, Docker.
    \end{itemize}}
\cvitem{Monitoring/logging}{ELK.}
\cvitem{Networking}{Practical understanding of TCP/IP networks, troubleshooting, DNS.}

\section{Trainings and certifications}
\cvitem{Admin}{Red Hat Certified System Administrator.}
\cvitem{Embedded}{Designing Embedded Systems with Yocto (Doulos).}
\cvitem{Training}{Train the trainers (Nokia) - 5 days long training for people providing trainings for others.}

%\ifdefstring{\lang}{pl}{
    \cfoot{\footnotesize{I hereby agree for processing the following personal information strictly for %
    the purposes of job recruitment \newline in accordance with the regulation for the protection of %
    personal data passed on 29.08.1997, Dz.U nr 133 poz. 883.}\newline\hspace{1in}{\tiny{\shortdate\today}}}
%}{
    %   \cfoot{\tiny{\shortdate\today}}
%}

\end{document}
